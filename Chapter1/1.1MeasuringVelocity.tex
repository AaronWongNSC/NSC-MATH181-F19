\documentclass{ximera}
%% handout
%% nohints
%% space
%% newpage
%% numbers


\prerequisites{none}

\title{1.1 How do we measure velocity?}

\begin{document}
\begin{abstract}
We interpret average velocity as a slope of a position function over a time interval.
\end{abstract}
\maketitle

\section{Overview}

We use link the ideas of slope and \emph{average velocity} of a moving object, and then begin to think about how we move from the notion of average velocity to that of \emph{instantaneous velocity}.  The idea of instantaneous velocity is the foundational idea of differential calculus, our main topic of study for the coming semester.


\section{Basic learning objectives}

These are the tasks you should be able to perform with reasonable fluency \textbf{when you arrive at our next class meeting}. Important new vocabulary words are indicated \emph{in italics}. 

\begin{itemize}
	\item Quickly plot a quadratic function of the form $s(t) = at^2 + bt + c$ both by hand and using appropriate technology.
	\item Determine the equation of a line through two points $(x_1, y_1)$ and $(x_2, y_2)$.
	\item Compute the \emph{average velocity} over an interval $[a,b]$ of a moving object with position function $s$.
	\item Identify the average velocity over a given interval as the slope of a certain line that correlates to a graph of the position function $s$.
\end{itemize}

\section{Advanced learning objectives}

In addition to mastering the basic objectives, here are the tasks you should be able to perform \textbf{after class, with practice}: 

\begin{itemize}
	\item Use multiple computations of average velocity on shrinking intervals to estimate instantaneous velocity.
	\item Understand how instantaneous velocity is thought of as a \emph{limit} of average velocities. 
\end{itemize}

\section{Resources}

\noindent
\emph{Reading}: Read pp.~4--6. \\

\noindent
\emph{Watching}: Here are some additional resources that have been developed to support your learning: 

\begin{itemize}
	\item Screencast 1.1.1: \youtube{http://gvsu.edu/s/qd}
	\item Screencast 1.1.2: \youtube{http://gvsu.edu/s/qe}
	\item Screencast 1.1.3: \youtube{http://gvsu.edu/s/ql}
\end{itemize}

\subsection*{Questions}

\noindent Complete each question below.  

\begin{question} 
The following questions concern the position function given by $s(t) = 64 - 16(t-1)^2$, which is the same function considered in Preview Activity $1.1$.

Compute the average velocity of the ball on each of the following time intervals:  
\begin{enumerate}
\item AV$_{[0.4,0.8]}=$ \answer{12.8} ft/s
\item AV$_{[0.7,0.8]}=$ \answer{8} ft/s
\item AV$_{[0.79,0.8]}=$ \answer{6.56} ft/s
\item AV$_{[0.799,0.8]}=$ \answer{6.416} ft/s 
\end{enumerate}

On a piece of scratch paper, sketch the line that passes through the points $A=(0.4, s(0.4))$ and $B=(0.8, s(0.8))$.  What is the meaning of the slope of this line?  
\begin{solution}
\begin{hint}
Look at AV$_{[0.4,0.8]}$ from the previous question. 
\end{hint}
\begin{freeResponse}
The slope is the average rate of change over the interval $[0.4,0.8]$.
\end{freeResponse}
\end{solution}

Use a graphing utility to plot the graph of $s(t) = 64 - 16(t-1)^2$ on an interval containing the value $t = 0.8$.  Then, zoom in repeatedly on the point $(0.8, s(0.8))$.  What do you observe about how the graph appears as you view it more and more closely?  
\begin{solution}
\begin{hint}
Go to \link{https://www.desmos.com/calculator/skkzi1dsll} to investigate.
\end{hint}
\begin{freeResponse}
If you zoom in enough it looks like a straight line with slope $m=6.4$.
\end{freeResponse}
\end{solution}
\end{question}

\begin{question}
Write a sentence in your own words that explains how we can estimate the instantaneous velocity of a moving object at a particular time by using average velocities on certain nearby time intervals.

\begin{freeResponse}
The instantaneous velocity is very close to the average velocity over a very short period of time.
\end{freeResponse}
\end{question}




\end{document}

